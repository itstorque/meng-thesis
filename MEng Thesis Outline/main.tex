\documentclass{article}
\usepackage[utf8]{inputenc}
% formatting imports
\usepackage{graphicx}
\usepackage[margin=1cm]{caption}
\usepackage[a4paper, total={6in, 8in}]{geometry}
\usepackage[section]{placeins}
\usepackage{array}

\usepackage{setspace}

% math imports
\usepackage{siunitx}
\usepackage{amsmath}
\usepackage{amsthm}
\usepackage{amssymb}
\usepackage{mathtools}
\usepackage{bbold}

\usepackage{todonotes}
\usepackage{longtable}

\usepackage[sorting=none]{biblatex}
\addbibresource{refs.bib}

\usepackage[fontsize=12pt]{fontsize}

\newcommand{\del}[2]{\frac{\partial #1}{\partial #2}}
\newcommand{\delsquare}[2]{\frac{\partial^2 #1}{\partial #2 ^2}}

\newcommand{\full}[2]{\frac{d #1}{d #2}}
\newcommand{\fullsquare}[2]{\frac{d^2 #1}{d #2 ^2}}

\newcommand{\ccf}[1]{`\textsf{#1}'}
\newcommand{\cf}[1]{\textsf{#1}}

\title{IGNORE THIS PAGE IN CURRENT VERSION}

\DeclareUnicodeCharacter{2212}{-}
\begin{document}

\doublespacing

% \maketitle

% \begin{center}
%     Massachusetts Institute of Technology \\
%     Department of Electrical Engineering and Computer Science
% \end{center}

% \begin{center}
%     Proposal for Thesis Research in Partial Fulfillment\\ of the Requirements for the Degree of\\
% Masters of Engineering
% \end{center}

% \vfill

% \begin{tabular}{rl}
% \textbf{Title:} & \textbf{Defect Identification in Superconducting 2-Terminal Devices} \\
% \\
% \textbf{Submitted by:} & T. Dandachi \\
%               & 69 Chestnut St. \\
%               & Cambridge, MA 02139
% \end{tabular}

% \vspace{15mm}
% \begin{tabular}{@{}p{1.5in}p{4in}@{}}
% \textbf{Signature of author:} & \hrulefill \\
% \end{tabular}

% \vfill

% \textbf{Expected Date of Completion:} May 2022

% \vfill

% \textbf{Laboratory:} Quantum Nanostructures and Nanofabrication under the Research Laboratory for Electronics

% \vfill

% \textbf{Abstract:}

% Simulating the operation of a superconducting device

% \vfill

% \textbf{Supervision Agreement:}

% The program outlined in this proposal is adequate for a Master's thesis. The supplies and facilities required are available, and I am willing to supervise the research and evaluate the thesis report.

% \vspace{15 mm}
% \begin{tabular}{@{}p{1.5in}p{4in}@{}}
%   & \hrulefill \\
% & K. K. Berggren, Prof. of Elec. Eng.
% \end{tabular}

% \newpage

\tableofcontents

\newpage

% figures are here: https://www.icloud.com/keynote/094kHrSTQlvohxqCHGUmZJXcg#thesis

\section{Introduction}

\subsection{Non-linearity in superconducting nanowires}

Inductance

Switching

\subsection{Nanowire Elements}

\subsubsection{SNSPDs}

\subsubsection{SNSPIs}

\subsubsection{Tapers}

\subsubsection{hTron}

\subsection{Problems Simulating Nanowires}

\section{spice-daemon --- a Python wrapper for SPICE solvers}

% spice-daemon and qnn-spice



\subsection{SPICE}

One popular way of simulating electronics is using SPICE 
(Simulation Program with Integrated Circuit Emphasis) first developed
at University of California, Berkley in the early 70s. 
SPICE solvers are particularly useful as they combine DC analysis (also known as
operating point analysis), AC analysis (linear small-signal frequency domain analysis) and 
transient analysis (time-domain large-signal solution of nonlinear differential algebraic equations)
among other analysis methods.\\

Since then, Berkley SPICE
inspired multiple other SPICE solvers including LTspice.
\todo{more on spice and ref} \todo{maybe FDM?}
LTspice is a popular free circuit simulator that is widely used \todo{ref}. SPICE models for 
superconducting electronics exist \todo{nanowire, hTron, nTron...}. 

\subsubsection{Netlists and Schematic Capture}

Interfacing with SPICE software involves generating a netlist --- a code snippet that defines
how the different circuit elements are connected to each other. Netlists have a \ccf{.net} (and 
sometimes a \ccf{.cir}) extension and can be used across different SPICE implementations. 
Netlists are encoded as ascii files and as such editing them is straightforward.\\

The netlist's syntax \todo{should i do this?}\\

Some commercial version of SPICE
software, such as LTspice, add Schematic Capture capability. Schematic Capture allows for a
native GUI encoding of a circuit to be converted into a netlist (in LTspice, that is a 
schematic file with the extension \ccf{.asc}).\\

\subsubsection{Output}

\subsubsection{Models}

ASY + lib

Saving Location

\subsubsection{Dependent Sources}

\subsubsection{Transient Simulation}

Problem/Why?

Hard to simulate effects

Fitting to data

Complicated analysis toolkits

Noise/custom inputs

Device level modelling

Solution/What?

Wrapper for spice solvers like LTspice.



\subsection{QNN SPICE}

In a collaborative setup where SPICE models might be edited (either continuously or infrequent small fixes,) 
having the ability to track the version of the models is important. One solution is to include a version 
string that the editor updates between revisions. Doing so however doesn't handle merge conflicts natively and
doesn't track file differences. From these requirements, the widely used version tracking software git can be
used to track the file differences and users need to always re-download the latest version of the model.

Ideally, every time a model file is downloaded, it gets placed in LTspice's library folder that contains all the base 
models. This process needs to be repeated for each model and it becomes tedious. The other alternative is have
the models all live in the same directory as your circuits and each model you use be downloaded manually into
that directory. The side-effect is you don't need to update all the models every time, however, the directory becomes
cluttered quickly and you will need multiple copies of every model on your system.

This is where qnn-spice comes in, MIT's Quantum Nanostructures and Nanofabrication group (QNN) has multiple
repositories, each with multiple spice models. By having a single repository track every repository containing
SPICE models, a single repository could track all the changes across every model produced by the QNN group. 
This single repository method takes advantage of git submodules, which tracks the head of each sub-repository.
A helper update script pulls every submodule and creates symbolic links in LTspice's library folder to each model.
The model library and symbol files to be included are specified in a YAML file -- a human-readable data-
serialization language.

The use of symbolic links means if a user edits the model in the cloned repository, LTspice sees the updated file.
When the update script pulls the main and sub repositories, the previous symbolic links are deleted and new ones are 
made. The sub-repository structure is copied into two \cf{qnn-spice} folders are created in the \cf{sub/} and \cf{sym/} 
subfolders of LTspice's library folder. 

The YAML file maps the path of each file to include in the repositories to a destination path in the two
\cf{qnn-spice} subfolders based on their extension. \todo{how to make a custom module grouping?}

\todo[inline]{Note for updates: lib files automatically updated. Schematic Capture related files (asy) aren't updated.}

\todo[inline]{Include library location?}

\subsection{Updating models using spice-daemon}

The main input for spice-daemon is a YAML file that defines simulation parameters, spice-daemon models and
toolkits. A YAML file can also be version tracked, allowing all parameterizations to be known by the host
python script.\todo[]{cool because u can save runs behind hidden menus and monte carlo things}

LTspice generates multiple files during every run, including a log file, a netlist file, an optional 
operating point analysis raw binary file and a raw binary file that includes the code of the simulation 
that is run. spice-daemon tracks the edit history of the log file, a YAML specification file and the 
circuit schematic using the \cf{WatchDog} object.

Every \cf{Simulation} object defines a couple of important \cf{File} objects that are always present regardless
of the user's setup for LTspice. \cf{File}s are an extension of python's \cf{Path} object that can additionally:
\begin{enumerate}
    \item track edit timelines,
    \item detect LTspice-native file encodings,
    \item generate dictionaries from YAML files, and
    \item read/write to files.
\end{enumerate}

The \cf{WatchDog} module periodically checks for edits on a \cf{Simulation}'s \cf{watch\_files},
a set of files that indicate a need to regenerate some (or all) spice-daemon produced files.
For instance, if someone edits an attribute for a component in the YAML specification file, 
the component library file needs to be regenerated to reflect the change in the attribute.

\todo[inline]{note on do not delete files after closing LTspice on Mac checkbox?}

\todo[inline]{setting up a spice-daemon simulation}

\subsection{Dynamic Models}

LTspice components are parametrizable using a constant global parameter space that can be used when
math expressions are being evaluated (such as the output voltage of a behavioural source or the inductance 
of an inductor). spice-daemon adds the ability to parameterize components beyond expressions by granting the
ability to edit a PWL file and the netlist of the model between runs.

\subsubsection{Tapers}

\subsubsection{Generating Noise}

\subsection{Arbitrary S$_{xy}$ models}

Diagram of 2port and n-port Sxy networks

Inductor, Cap model for T.D.

\subsubsection{2-port model}

E.g. Tapers!

\subsubsection{n-port model}

E.g. Bias Tee

\subsection{Postprocessing/Toolkits}

\section{Model Stability}

\subsection{Stability in F.D.M.}

\subsubsection{Stabiltiy in LTSpice}

\subsection{Malicious Circuits}

Why? How? (tline timestepping with half res?)

\subsubsection{Proof of Equivalence to Stability}

\subsection{Better nanowire models}

\subsubsection{Current nanowire model}

Layout of in-depth model

\subsubsection{S.A. of current nanowire model}

\subsubsection{Different Integrator}

\subsubsection{1 Element Models}

\subsubsection{0 Resistance Models}

\section{Efficient Simulation}

Julia simulator

\subsection{Tline Model}

\subsubsection{Equivalent Circuit}

\subsubsection{Kernel}

\subsubsection{GPU}

\subsection{Harmonic Balance}

\subsubsection{TD Assist}

\subsection{Device Symmetries}

\subsection{Coupling Diff. Eq. (or Thermal Model?)}

Maybe sections should be separate. Phonon-Electron transport.

Complex electric coupling

\subsubsection{TDC}

\subsubsection{SNSPI coupling}

\subsection{Precomputation}

\subsection{ML Optimization}

\subsubsection{Symbolic Solver}

\subsubsection{Tapers}

Note on resistive groups? -> Sonnet?

\subsubsection{Differentiable Simulator}

\subsubsection{Inverse design}

\subsubsection{Monte Carlo Simulation}

\section{Time Domain Reflectometry}

\subsection{Idea}

\subsection{Optimal Search Theory}

\subsection{Simulation}

\subsection{Experimental Setup}

\subsection{Results}

\subsection{Future Ideas}

\end{document}
