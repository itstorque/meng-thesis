\documentclass{article}
\usepackage[utf8]{inputenc}
% formatting imports
\usepackage{graphicx}
\usepackage[margin=1cm]{caption}
\usepackage[a4paper, total={6in, 8in}]{geometry}
\usepackage[section]{placeins}
\usepackage{array}

% math imports
\usepackage{siunitx}
\usepackage{amsmath}
\usepackage{amsthm}
\usepackage{amssymb}
\usepackage{mathtools}
\usepackage{bbold}

\usepackage{todonotes}
\usepackage{longtable}

\usepackage[sorting=none]{biblatex}
\addbibresource{refs.bib}

\newcommand{\del}[2]{\frac{\partial #1}{\partial #2}}
\newcommand{\delsquare}[2]{\frac{\partial^2 #1}{\partial #2 ^2}}

\newcommand{\full}[2]{\frac{d #1}{d #2}}
\newcommand{\fullsquare}[2]{\frac{d^2 #1}{d #2 ^2}}

\title{IGNORE THIS PAGE IN CURRENT VERSION}

\DeclareUnicodeCharacter{2212}{-}
\begin{document}

\maketitle

\begin{center}
    Massachusetts Institute of Technology \\
    Department of Electrical Engineering and Computer Science
\end{center}

\begin{center}
    Proposal for Thesis Research in Partial Fulfillment\\ of the Requirements for the Degree of\\
Masters of Engineering
\end{center}

\vfill

\begin{tabular}{rl}
\textbf{Title:} & \textbf{Defect Identification in Superconducting 2-Terminal Devices} \\
\\
\textbf{Submitted by:} & T. Dandachi \\
              & 69 Chestnut St. \\
              & Cambridge, MA 02139
\end{tabular}

\vspace{15mm}
\begin{tabular}{@{}p{1.5in}p{4in}@{}}
\textbf{Signature of author:} & \hrulefill \\
\end{tabular}

\vfill

\textbf{Expected Date of Completion:} May 2022

\vfill

\textbf{Laboratory:} Quantum Nanostructures and Nanofabrication under the Research Laboratory for Electronics

% \vfill

% \textbf{Abstract:}

% Simulating the operation of a superconducting device

\vfill

\textbf{Supervision Agreement:}

The program outlined in this proposal is adequate for a Master's thesis. The supplies and facilities required are available, and I am willing to supervise the research and evaluate the thesis report.

\vspace{15 mm}
\begin{tabular}{@{}p{1.5in}p{4in}@{}}
  & \hrulefill \\
& K. K. Berggren, Prof. of Elec. Eng.
\end{tabular}

\newpage

\tableofcontents

\section{Introduction}

\subsection{Non-linearity in superconducting nanowires}

\subsection{Nanowire Elements}

\subsubsection{SNSPDs}

\subsubsection{SNSPIs}

\subsubsection{Tapers}

\subsubsection{hTron}

\subsection{Problems Simulating Nanowires}

\section{SPICE}

spice-daemon and qnn-spice

\subsection{QNN SPICE}

\subsection{Dynamic Models}

E.g. Tapers

\subsection{Generating Noise}

\subsection{Arbitrary S$_{xy}$ models}

\subsubsection{2-port model}

E.g. Tapers!

\subsubsection{n-port model}

E.g. Bias Tee

\subsection{Postprocessing}

\section{Model Stability}

\subsection{Stability in F.D.M.}

\subsubsection{Stabiltiy in LTSpice}

\subsection{Malicious Circuits}

Why? How? (tline timestepping with half res?)

\subsubsection{Proof of Equivalence to Stability}

\subsection{Better nanowire models}

\subsubsection{Current nanowire model}

Layout of in-depth model

\subsubsection{S.A. of current nanowire model}

\subsubsection{Different Integrator}

\subsubsection{1 Element Models}

\subsubsection{0 Resistance Models}

\section{Efficient Simulation}

Julia simulator

\subsection{Tline Model}

\subsubsection{Equivalent Circuit}

\subsubsection{Kernel}

\subsubsection{GPU}

\subsection{Harmonic Balance}

\subsubsection{TD Assist}

\subsection{Device Symmetries}

\subsection{Coupling Diff. Eq. (or Thermal Model?)}

Maybe sections should be separate. Phonon-Electron transport.

Complex electric coupling

\subsubsection{TDC}

\subsubsection{SNSPI coupling}

\subsection{Precomputation}

\subsection{ML Optimization}

\subsubsection{Symbolic Solver}

\subsubsection{Tapers}

Note on resistive groups? -> Sonnet?

\subsubsection{Differentiable Simulator}

\subsubsection{Inverse design}

\subsubsection{Monte Carlo Simulation}

\section{Time Domain Reflectometry}

\subsection{Idea}

\subsection{Optimal Search Theory}

\subsection{Simulation}

\subsection{Experimental Setup}

\subsection{Results}

\subsection{Future Ideas}

\end{document}
