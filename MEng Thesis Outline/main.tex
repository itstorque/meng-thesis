\documentclass{article}
\usepackage[utf8]{inputenc}
% formatting imports
\usepackage{graphicx}
\usepackage[margin=1cm]{caption}
\usepackage[a4paper, total={6in, 8in}]{geometry}
\usepackage[section]{placeins}
\usepackage{array}

\usepackage{setspace}

% math imports
\usepackage{siunitx}
\usepackage{amsmath}
\usepackage{amsthm}
\usepackage{amssymb}
\usepackage{mathtools}
\usepackage{bbold}

\usepackage{todonotes}
\usepackage{longtable}

\usepackage[sorting=none]{biblatex}
\addbibresource{refs.bib}

\usepackage[fontsize=12pt]{fontsize}

\newcommand{\del}[2]{\frac{\partial #1}{\partial #2}}
\newcommand{\delsquare}[2]{\frac{\partial^2 #1}{\partial #2 ^2}}

\newcommand{\full}[2]{\frac{d #1}{d #2}}
\newcommand{\fullsquare}[2]{\frac{d^2 #1}{d #2 ^2}}

\newcommand{\ccf}[1]{`\textsf{#1}'}
\newcommand{\cf}[1]{\textsf{#1}}

\title{IGNORE THIS PAGE IN CURRENT VERSION}

\DeclareUnicodeCharacter{2212}{-}
\begin{document}

\doublespacing

% \maketitle

% \begin{center}
%     Massachusetts Institute of Technology \\
%     Department of Electrical Engineering and Computer Science
% \end{center}

% \begin{center}
%     Proposal for Thesis Research in Partial Fulfillment\\ of the Requirements for the Degree of\\
% Masters of Engineering
% \end{center}

% \vfill

% \begin{tabular}{rl}
% \textbf{Title:} & \textbf{Defect Identification in Superconducting 2-Terminal Devices} \\
% \\
% \textbf{Submitted by:} & T. Dandachi \\
%               & 69 Chestnut St. \\
%               & Cambridge, MA 02139
% \end{tabular}

% \vspace{15mm}
% \begin{tabular}{@{}p{1.5in}p{4in}@{}}
% \textbf{Signature of author:} & \hrulefill \\
% \end{tabular}

% \vfill

% \textbf{Expected Date of Completion:} May 2022

% \vfill

% \textbf{Laboratory:} Quantum Nanostructures and Nanofabrication under the Research Laboratory for Electronics

% \vfill

% \textbf{Abstract:}

% Simulating the operation of a superconducting device

% \vfill

% \textbf{Supervision Agreement:}

% The program outlined in this proposal is adequate for a Master's thesis. The supplies and facilities required are available, and I am willing to supervise the research and evaluate the thesis report.

% \vspace{15 mm}
% \begin{tabular}{@{}p{1.5in}p{4in}@{}}
%   & \hrulefill \\
% & K. K. Berggren, Prof. of Elec. Eng.
% \end{tabular}

% \newpage

\tableofcontents

\newpage

% figures are here: https://www.icloud.com/keynote/094kHrSTQlvohxqCHGUmZJXcg#thesis

\section{Introduction}

\subsection{Non-linearity in superconducting nanowires}

Inductance

Switching

\subsection{Nanowire Elements}

\subsubsection{SNSPDs}

\subsubsection{SNSPIs}

\subsubsection{Tapers}

\subsubsection{hTron}

\subsection{Problems Simulating Nanowires}

\section{spice-daemon --- a Python wrapper for SPICE solvers}

% spice-daemon and qnn-spice



\subsection{SPICE}

One popular way of simulating electronics is using SPICE 
(Simulation Program with Integrated Circuit Emphasis) first developed
at University of California, Berkley in the early 70s. 
SPICE solvers are particularly useful as they combine DC analysis (also known as
operating point analysis), AC analysis (linear small-signal frequency domain analysis) and 
transient analysis (time-domain large-signal solution of nonlinear differential algebraic equations)
among other analysis methods.\\

Since then, Berkley SPICE
inspired multiple other SPICE solvers including LTspice.
\todo{more on spice and ref} \todo{maybe FDM?}
LTspice is a popular free circuit simulator that is widely used \todo{ref}. SPICE models for 
superconducting electronics exist \todo{nanowire, hTron, nTron...}. 

\subsubsection{Netlists and Schematic Capture}

Interfacing with SPICE software involves generating a netlist --- a code snippet that defines
how the different circuit elements are connected to each other. Netlists have a \ccf{.net} (and 
sometimes a \ccf{.cir}) extension and can be used across different SPICE implementations. 
Netlists are encoded as ascii files and as such editing them is straightforward.\\

The netlist's syntax \todo{should i do this?}\\

Some commercial version of SPICE
software, such as LTspice, add Schematic Capture capability. Schematic Capture allows for a
native GUI encoding of a circuit to be converted into a netlist (in LTspice, that is a 
schematic file with the extension \ccf{.asc}).\\

\subsubsection{Output}

\subsubsection{Models}

ASY + lib

Saving Location

\subsubsection{Dependent Sources}

\subsubsection{Transient Simulation}

Problem/Why?

Hard to simulate effects

Fitting to data

Complicated analysis toolkits

Noise/custom inputs

Device level modelling

Solution/What?

Wrapper for spice solvers like LTspice.



\subsection{QNN SPICE}

In a collaborative setup where SPICE models might be edited (either continuously or infrequent small fixes,) 
having the ability to track the version of the models is important. One solution is to include a version 
string that the editor updates between revisions. Doing so however doesn't handle merge conflicts natively and
doesn't track file differences. From these requirements, the widely used version tracking software git can be
used to track the file differences and users need to always re-download the latest version of the model.

Ideally, every time a model file is downloaded, it gets placed in LTspice's library folder that contains all the base 
models. This process needs to be repeated for each model and it becomes tedious. The other alternative is have
the models all live in the same directory as your circuits and each model you use be downloaded manually into
that directory. The side-effect is you don't need to update all the models every time, however, the directory becomes
cluttered quickly and you will need multiple copies of every model on your system.

This is where qnn-spice comes in, MIT's Quantum Nanostructures and Nanofabrication group (QNN) has multiple
repositories, each with multiple spice models. By having a single repository track every repository containing
SPICE models, a single repository could track all the changes across every model produced by the QNN group. 
This single repository method takes advantage of git submodules, which tracks the head of each sub-repository.
A helper update script pulls every submodule and creates symbolic links in LTspice's library folder to each model.
The model library and symbol files to be included are specified in a YAML file -- a human-readable data-
serialization language.

The use of symbolic links means if a user edits the model in the cloned repository, LTspice sees the updated file.
When the update script pulls the main and sub repositories, the previous symbolic links are deleted and new ones are 
made. The sub-repository structure is copied into two \cf{qnn-spice} folders are created in the \cf{sub/} and \cf{sym/} 
subfolders of LTspice's library folder. 

The YAML file maps the path of each file to include in the repositories to a destination path in the two
\cf{qnn-spice} subfolders based on their extension. \todo{how to make a custom module grouping?}

\todo[inline]{Note for updates: lib files automatically updated. Schematic Capture related files (asy) aren't updated.}

\todo[inline]{Include library location?}

\subsection{Updating models using spice-daemon}

The main input for spice-daemon is a YAML file that defines simulation parameters, spice-daemon models and
toolkits. A YAML file can also be version tracked, allowing all parameterizations to be known by the host
python script.\todo[]{cool because u can save runs behind hidden menus and monte carlo things}

LTspice generates multiple files during every run, including a log file, a netlist file, an optional 
operating point analysis raw binary file and a raw binary file that includes the code of the simulation 
that is run. spice-daemon tracks the edit history of the log file, a YAML specification file and the 
circuit schematic using the \cf{WatchDog} object.

Every \cf{Simulation} object defines a couple of important \cf{File} objects that are always present regardless
of the user's setup for LTspice. \cf{File}s are an extension of python's \cf{Path} object that can additionally:
\begin{enumerate}
    \item track edit timelines,
    \item detect LTspice-native file encodings,
    \item generate dictionaries from YAML files, and
    \item read/write to files.
\end{enumerate}

The \cf{WatchDog} module periodically checks for edits on a \cf{Simulation}'s \cf{watch\_files},
a set of files that indicate a need to regenerate some (or all) spice-daemon produced files.
For instance, if someone edits an attribute for a component in the YAML specification file, 
the component library file needs to be regenerated to reflect the change in the attribute.

\todo[inline]{note on do not delete files after closing LTspice on Mac checkbox?}

\todo[inline]{setting up a spice-daemon simulation}

\todo[inline]{Diagram of the listen and write files}

\subsection{Dynamic Models}

LTspice components are parametrizable using a constant global parameter space that can be used when
math expressions are being evaluated (such as the output voltage of a behavioural source or the inductance 
of an inductor). spice-daemon adds the ability to parameterize components beyond expressions by granting the
ability to edit a PWL file and the netlist of the model between runs.
\todo[]{note on all libs being merged into one!}

\subsubsection{Lumped Element Transmission Lines and Tapers}

One type of dynamic model that is incorporated into spice-daemon are lumped 
element transmission lines. Instead of using LTspice's built-in transmission
line models (either the Lossless Transmission Lines (T elements) or the 
Lossy Transmission Lines (O elements)), spice-daemon allows you to specify a
lumped-element version. 

The Lossless Transmission Line model has a bunch of limitations: it models only one propagation mode,
 doesn't support non-linear response functions and doesn't 
model the DC behaviour correctly. The Lossy Transmission Line also suffers from multiple caveats,
it doesn't support frequency dependence for loss and it also doesn't support non-linear response 
functions. %https://ltwiki.org/index.php?title=O-device_(Lossy_Transmission_Line)_and_T-device_(Lossless_Transmission_Line)_modelling_issues https://ltwiki.org/files/LTspiceHelp.chm.html

For well-defined behaviour with non-convolution based models, it is helpful to be able to run
a lumped element model from within LTspice. However, this would involve laying down thousands 
of repeating chunks of elements manually. One use of dynamic models is generating a model 
that encodes variable length logic. In this method of programming a lumped element transmission
line, by changing one parameter in the configuration file, we are able to affect the circuit
topology and add a large number of elements - this type of automation is not possible without
some sort of external daemon.

One extension to this one-to-many mapping for the transmission line is our ability to program
lumped-element tapers in. The transmission line models in LTspice work for lines with constant 
parameters (impedance, propagation velocity, loss etc.). With a lumped element model that is fully
controlled by spice-daemon, changing the impedance of one port can map the inductance and capacitance
of each finite element to a pair of values based on the taper geometry chosen. This adds another layer
of abstraction where we can define an impedance matched transmission line with a klopfenstein geometry
between two impedances $Z_{in}, Z_{out}$. If we change $Z_{in}$, the spice-daemon instance calculates
new tapering parameters smoothly perturbing the impedance of the line from $Z_{in}$ to $Z_{out}$ and
updates the library file for the taper element. When LTspice runs a new simulation, it pulls the latest
lib file with the new impedance matched taper.

\subsubsection{Generating Noise}

\todo[]{why this is important for simulating nanowires}

Noise analysis in LTspice is limited, especially on non-linear systems. Since superconducting electronics
are highly non-linear, it is essential that we run all our analysis in the transient analysis mode (time-basis 
small-signal AC simulation). In this mode, we can use a voltage source with an LTspice native math command
to generate noise such as \cf{noise}, \cf{random}, \cf{gaussian}, \cf{white}. However, these noise commands
generate noise that is correlated amongst instances and is not gaussian in nature. One workaround that was
discovered by the LTspice community was using 4 voltage sources each producing a shifted seed value for 
gaussian noise that guarantees that the seed doesn't overlap with the simulation time. By concatenating the
outputs of the behavioural voltage sources, the central limit theorem makes the noise distribution behave more 
gaussian. Note that the number 4 was picked due to a trade-off between the complexity of generating and simulating
that noise and how gaussian it is -- the more sources there are, the more gaussian the noise distribution is.

One other workaround is to use PWL files. PWL files allow you to input piece-wise linear functions into LTspice
sources that are not necessarily behavioural. The PWL opeartion mode maps (time, value) pairs to a continuous 
output value based on the simulation time. However, if a simulation has $N$ points it is solved at and you 
provide $N$ noise points, then there is no extrapolation that occurs, provided these points are chosen at random 
from a gaussian distribution, then the voltage source will generate noise that has a gaussian probability density
function. The process of re-creating this noise file and making sure there is enough data points and that the time
axes of the simulation and the noise data is matched is tedious.

To solve this issue, spice-daemon can handle the creation of noise sources and their accompanying PWL files,
abstracting them behind one symbol file. The user defines a noise source type (voltage or current), the 
noise distribution it should follow (poisson, gaussian, 1 over f, etc.) and distribution parameters (mean, 
standard deviation, etc.). The spice-daemon instance then generates a symbol file for a noise source that 
references a separate library sub-circuit for each noise instance. The sub-circuits for each noise source 
instance references a separate PWL file. Each PWL file has (time, value) pairs generated in python using
numpy. As a result, we know that the noise inputs to LTspice actually follow a certain distribution, and we
can verify that the correct noise distribution is being simulated inside LTspice.

This method also allows us to easily have multiple non-correlated noise sources, as well as noise 
distributions that aren't pseudo-gaussian or pseudo-uniform. The scaling and math required to generate
LTspice native gaussian noise for multiple sources was unfeasible in terms of simulation time as well
as calculating seed values off of the simulation time to gaurantee there was no time correlation between
the sources.


\subsection{Arbitrary S$_{xy}$ models}

Diagram of 2port and n-port Sxy networks

Inductor, Cap model for T.D.

\subsubsection{2-port model}

E.g. Tapers!

\subsubsection{n-port model}

E.g. Bias Tee

\subsection{Postprocessing/Toolkits}

\section{Model Stability}

\subsection{Stability in F.D.M.}

\subsubsection{Stabiltiy in LTSpice}

\subsection{Malicious Circuits}

Why? How? (tline timestepping with half res?)

\subsubsection{Proof of Equivalence to Stability}

\subsection{Better nanowire models}

\subsubsection{Current nanowire model}

Layout of in-depth model

\subsubsection{S.A. of current nanowire model}

\subsubsection{Different Integrator}

\subsubsection{1 Element Models}

\subsubsection{0 Resistance Models}

\section{Efficient Simulation}

Julia simulator

\subsection{Tline Model}

\subsubsection{Equivalent Circuit}

\subsubsection{Kernel}

\subsubsection{GPU}

\subsection{Harmonic Balance}

\subsubsection{TD Assist}

\subsection{Device Symmetries}

\subsection{Coupling Diff. Eq. (or Thermal Model?)}

Maybe sections should be separate. Phonon-Electron transport.

Complex electric coupling

\subsubsection{TDC}

\subsubsection{SNSPI coupling}

\subsection{Precomputation}

\subsection{ML Optimization}

\subsubsection{Symbolic Solver}

\subsubsection{Tapers}

Note on resistive groups? -> Sonnet?

\subsubsection{Differentiable Simulator}

\subsubsection{Inverse design}

\subsubsection{Monte Carlo Simulation}

\section{Time Domain Reflectometry}

\subsection{Idea}

\subsection{Optimal Search Theory}

\subsection{Simulation}

\subsection{Experimental Setup}

\subsection{Results}

\subsection{Future Ideas}

\end{document}
